\documentclass{article}

\begin{document}
	
\begin{center}
	
	\underline{\textbf{\Huge 1er Proyecto de}}\\
	\vspace{0.5cm}	
	\underline{\textbf{\Huge Programación: Moogle!!}}\\
	
	\vspace{3.3cm}
	
	\textbf{\Huge Diego Hernández}\\
	\vspace{0.5cm}	
	\textbf{\Huge Rodríguez}\\
	
	\vspace{2.3cm}
	
	\underline{\Huge C113}\\
	
	\vspace{3.3cm}
	
	\underline{\Huge July 21st, 2023}\\
	
\end{center}

\newpage

\Large Entre las principales características de mi proyecto, existen algunas como el uso del modelo vectorial de búsqueda para determinar el resultado más óptimo para cada consulta.\\
\\
\Large Para el funcionamiento de dicho modelo hace falta calcular el \textbf{TF-IDF} de cada palabra de cada texto y de la consulta lo cual utilice las ecuaciones de:\\
\begin{equation}
\Large TF = 1 + \log{(wc)}
\end{equation}
\begin{equation}
\Large IDF = \log{(\frac{dc}{wdf})}
\end{equation}
\\
\textbf{\large Leyenda: \\
	\\
	\large wc : cantidad de veces que aparece una palabra en un texto\\
	\\
	\large dc : cantidad de documentos\\
	\\
	\large wdf: cantidad de documentos en los que aparece una palabra\\}
\\
\Large Nota: estos valores se refieren a los relativos a cada texto, ya que todo se calcula en función de cada documento dando en la gran mayoría de los casos valores distintos.
\\
\Large Luego utilizando esos valores se aplica un producto escalar entre vectores y se dividen por la raíz de los cuadrados de los valores de \textbf{TF-IDF} de cada palabra en el texto y en la consulta y así queda el valor del coseno necesario para devolver un buen resultado.



	
\end{document}